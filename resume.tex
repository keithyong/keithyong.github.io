\documentclass[margin]{res}
\usepackage{latexsym}
\usepackage{hyperref}
\usepackage{url}
\textwidth=5.2in        % The text width of the section headings.

\begin{document}
\name{Keith Yong\\[15pt]}

\address{{\bf Address}\\346 Nicholas Ct \\Wilmington, DE 19808\\(302) 887-0387}
\address{{\bf Links}\\\url{terda12@gmail.com}\\\url{github.com/keithyong}\\\url{keithy.me}}

\begin{resume}

\section{Education}
University of Delaware\\
Computer Science BS\\
Class of 2016\\

\section{Work}
\textit{Conferency (2015 - Present)} - Worked as a full stack web developer and a React specialist for a small startup. Built various features for a multipage web application on top of Flask. Worked on creating new REST API endpoints. Built various front-ends out of Bootstrap. Created CSS for the landing page and navbars. Used a Git workflow system.

\section{Projects}

Find my web projects at \url{keithy.me}

\textit{Reddit Clone (2015)}\\ Created a \texttt{reddit.com} clone. Used React as the primary front end framework, and created React components such as \texttt{Post.jsx} and \texttt{Comment.jsx} to avoid DRY. Designed a full SQL schema to handle posts, comments, upvotes, and other features. Used PostgresQL trigger functions to handle efficient upvoting. Utilized webpack to bundle up all dependencies into several \texttt{bundle.js} files. \underline{\texttt{\href{https://github.com/keithyong/pyramus}{Github}}} \underline{\texttt{\href{http://pyramus.keithy.me}{Demo}}}

\textit{Realtime Chat Room (February 2015)}\\ Wrote a real-time online chatroom using a React frontend, and socket.io, Node, and postgresQL backend. \underline{\texttt{\href{https://github.com/keithyong/chat-room}{Github}}} \underline{\texttt{\href{http://chatter.keithy.me}{Demo}}}

\textit{Pomodoro Timer (January 2015)}\\ Wrote an online pomodoro timer in pure Javascript and CSS with a focus on UX and design. Used CSS media queries to ensure a mobile ready responsive experience. \underline{\texttt{\href{https://github.com/keithyong/pomodoro}{Github}}} \underline{\texttt{\href{http://keithy.me/pomodoro}{Demo}}}


\textit{Online Fibonacci Calculator (December 2014)}\\ Wrote an online nth-Fibonacci calculator in Javascript. Gives the users a choice to calculate using dynamic programming or recursion, and notifies them of the running time. Wrote an explanation under the calculator which analyzes how dynamic programming significantly reduces running time. Used D3.js and media queries to make mobile responsive recursion tree visuals. \underline{\texttt{\href{https://github.com/keithyong/fibonacci-calc}{Github}}} \underline{\texttt{\href{http://keithy.me/fibonacci-calc}{Demo}}}

% \textit{Rat Dungeon (December 2014)}\\ Worked with a team to write a genetic algorithm framework in C++ that grooms a genetic rat to survive a dungeon for the most number of moves. Modeled algorithm after natural selection where the fittest rat moves to the next generation and biological reproduction where a portion of the male and a portion of the female's chromosomes are exchanged during gene crossover. Allows tweaking of different variables such as mutation chance and crossover rat to maximize rat improvement rate.

\section{Skills}
Good - Node.js, Express, React, PostgresQL, Javascript, CSS

Familiar - Flask, Python, C, C++, Java, LaTeX

\end{resume}

\end{document}
